\documentclass[letter]{article}
\usepackage[spanish]{babel}
\usepackage[margin=1in]{geometry}
\usepackage{amsmath}
\usepackage{amsthm}
\usepackage{amssymb}
\usepackage[utf8]{inputenc}
\usepackage{graphicx, color}
\usepackage{algorithm}
\usepackage{algpseudocode}
\usepackage{mathrsfs}


% Some definitions
\floatname{algorithm}{Algoritmo}

% Author info
\title {Secuencia más larga de vecinos que están ordenados}
\author{ Jorge Luis Esposito Albornoz$^1$  Juan Sebastián Herrera Guaitero$^1$}
\date{
	$^1$Departamento de Ingeniería de Sistemas, Pontificia Universidad Javeriana\\Bogotá,  Colombia \\
	\texttt{\{jesposito,jsebastianherrera\}@javeriana.edu.co}\\~\\
	\today
}

\begin{document}
\maketitle

\begin{abstract}
	En este documento se presenta la formalización del problema de encontrar la secuencia más larga de vecinos que están ordenados.
	\textbf{Palabras clave:} matriz, adyacentes.
\end{abstract}

\tableofcontents

\newpage
\section{Formalización problema}
El problema de encontrar la secuencia mas larga de vecinos que estan ordenados, es un problema muy utilizado en el mundo empresarial para realizar entrevistas de conocimiento
a los candidatos. Su dificultad particular radica en la solucion del problema, los primeros acercamientos a la solucion suelen ser a traves
de la recursion ingenua, sin embargo, utilizar programacion dinamica es una buena opcion.

\subsection{Definición del problema}
Dada una matriz cuadrada natural de tamaño $NxN$, que contiene los números únicos en el rango $[1,NxN]$, pero que no están forzosamente en orden, encontrar la secuencia más larga de vecinos que están ordenados y los elementos adyacentes en la matriz que tienen una diferencia de $+1$.
\break
\break
El problema de la secuencia mas larga de vecinos ordenados se define apartir de:
\begin{enumerate}
	\item Una matriz $M$ de elementos $a\in\mathbb{Z}$.
\end{enumerate}
Generar una secuencia $S$ cuyos elementos cumplan con una relacion $a<b$.
\begin{itemize}
	\item \textbf{Entradas:}
	      \begin{itemize}
		      \item $M = \left<\left<a_{ij}\in \mathbb{Z} \right>\right>$
	      \end{itemize}
	\item \textbf{Salidas:}
	      \begin{itemize}
		      \item $S = \left<e_i \in M\right> ~ | ~ \forall i \in M ~ e_i \le e_{i+1}  $
	      \end{itemize}
\end{itemize}
\section{Algoritmo de solucion}
\begin{algorithm}[!htb]
	\caption{Naive solution}
	\begin{algorithmic}[1]
		\Procedure{$memo$}{$M$}
		\If{$i = j$}
		\State $R ~ +\gets ~ ``A" ~ + ~ string(i+1)$
		\Else
		\State $R ~ +\gets ~ ``(" $
		\State $q \gets B_{ij}$
		\State \Call{MatrixChainBk}{$B,R,i,q$}
		\State \Call{MatrixChainBk}{$B,R,q+1,j$}
		\State $R ~ +\gets ``)"$
		\EndIf
		\EndProcedure
	\end{algorithmic}
\end{algorithm}
\subsection{Analisis de complejidad}
\subsection{Invariante}
\section{Analisis experimental}
\subsection{Protocolo}
\subsection{Procedimiento}
\subsection{Resultado}
\subsection{Analisis}
\section{Conclusiones}

\end{document}

